% Created 2015-08-31 一 22:12
\documentclass[11pt]{article}
\usepackage[utf8]{inputenc}
\usepackage[T1]{fontenc}
\usepackage{fixltx2e}
\usepackage{graphicx}
\usepackage{grffile}
\usepackage{longtable}
\usepackage{wrapfig}
\usepackage{rotating}
\usepackage[normalem]{ulem}
\usepackage{amsmath}
\usepackage{textcomp}
\usepackage{amssymb}
\usepackage{capt-of}
\usepackage{hyperref}
\usepackage{minted}
\usepackage{ctex}
\author{林懿伦}
\date{2015 年 8 月 31 日}
\title{C++ Primer 阅读指导}
\hypersetup{
 pdfauthor={林懿伦},
 pdftitle={C++ Primer 阅读指导},
 pdfkeywords={},
 pdfsubject={},
 pdfcreator={Emacs 25.0.50.1 (Org mode )},
 pdflang={English}}
\begin{document}

\maketitle
\tableofcontents


\section{Week 1 : C++基础入门}
\label{sec:orgheadline16}

\subsection{第1章 : 开始}
\label{sec:orgheadline9}

\subsubsection{章节导读}
\label{sec:orgheadline1}
本章通过一个示例程序入手,开始接触C++数据类型,变量,表达式及函数等基本概念。这一章仅对于完全没有编程基础的同学而言有一定难度,需要把每个代码示例完成并编译运行,查看输入与输出来理解程序的含义。

本章阅读过程中请完成下述任务:
\begin{enumerate}
\item 按照章节的介绍完成书中的示例程序,并充分理解程序的每一部分。
\item 程序完成后,使用g++进行编译并运行,同时查看运行的输出结果。
\end{enumerate}

\subsubsection{课后练习}
\label{sec:orgheadline7}
书中练习较多,本章建议完成练习1.3,1.11,1.15,1.23,1.25。

\begin{enumerate}
\item 练习1.3
\label{sec:orgheadline2}
\begin{minted}[mathescape=true,linenos=true,numbersep=5pt,frame=lines,framesep=2mm]{c++}
#include <iostream>
/*
 *Author: Гагарин
 *Date: 2015-08-31
 */
int main()
{
    std::cout << "Oh great user! Whats' your name, may I ask ?" << std::endl
              << "I'm: ";
    std::string user = "";
    std::cin >> user;
    std::cout << "Hello, " << user << "!" << std::endl;
    return 0;
}
\end{minted}

\item 练习1.11
\label{sec:orgheadline3}
\begin{minted}[mathescape=true,linenos=true,numbersep=5pt,frame=lines,framesep=2mm]{c++}
#include <iostream>

int main()
{
    int start = 0,  end = 0;
    std::cout << "Input two numbers, I'll return the sum of numbers between them." <<std::endl;
    std::cin >> start >> end;
    int ord_start = start;
    int ord_end = end;
    int sum = 0;
    while (start < end)
    {
        /*
         * an unefficient way to sum these numbers
         */
        ++start;
        --end;
        if(start != end)
        {
            std::cout << "That are " << start << "+" << end << " = " << start+end <<std::endl;
            sum += start+end ;
        }
        else
        {
            std::cout << "Left only one number: " << start <<std::endl;
            sum += start;
        }
       std::cout << "Sum is " << sum << std::endl;
    }
    std::cout << "The sum of numbers between " << ord_start << " and " << ord_end << " is " << sum << std::endl;
    return 0;
}
\end{minted}

\item 练习1.15
\label{sec:orgheadline4}
\begin{minted}[mathescape=true,linenos=true,numbersep=5pt,frame=lines,framesep=2mm]{c++}
#include <iostream>

int main()
{
    int a = '1'; // Type Error
    b = 2; // Decleartion Error
    std::cout << "a + b =" << std::endl //Syntax Error
}
\end{minted}
编译源代码
\begin{quote}
g++ test15.cpp -o error -Wall
\end{quote}
提示错误信息:
\begin{verbatim}
test15.cpp: In function ‘int main()’:
test15.cpp:6:5: error: ‘b’ was not declared in this scope
b = 2; // Decleartion Error
^

test15.cpp:8:1: error: expected ‘;’ before ‘}’ token
}
^
test15.cpp:5:9: warning: unused variable ‘a’ [-Wunused-variable]
int a = '1'; // Type Error
    ^
\end{verbatim}
注意, 并未提示变量 \texttt{a} 的类型错误.
\item 练习1.23
\label{sec:orgheadline5}

\item 练习1.25
\label{sec:orgheadline6}
\end{enumerate}

\subsubsection{核心知识点}
\label{sec:orgheadline8}
\begin{itemize}
\item 如何编译及运行一个简单的C++程序
\item 基本的输入与输出( \texttt{stdin} 与  \texttt{stdout} )及 \texttt{iostream} 库如何使用
\item C++的控制流( \texttt{if} , \texttt{for} , \texttt{while} )
\item C++的类与成员函数的基本概念,简单理解即可,后面章节会详细介绍
\item 简单的书店程序编写、编译与测试
\end{itemize}

\subsection{第2章 : 变量和基本类型}
\label{sec:orgheadline12}

\subsubsection{章节导读}
\label{sec:orgheadline10}

本章介绍C++中的数据类型,以及如何存储和操作数据。请按照书籍要求完成基本类型的学习并开始学习C++标准库里的复杂类型。

本章练习都非常基础,有助于对数据类型的理解,建议全部完成。

\subsubsection{核心知识点}
\label{sec:orgheadline11}

核心知识点包含:

\begin{itemize}
\item C++的内置类型的概念及使用场景:bool char short int long float double
\item \textbf{【难点】} \footnote{标注【难点】的内容需要花费更多时间仔细阅读理解。} 不同数据类型间类型转换的方法及预期结果
\item 作用域的概念
\item 如何定义和使用变量
\item \textbf{【难点】}  复合类型的概念与使用(引用,指针)
\item const 限定符的使用,const 的引用以及指针部分比较难理解
\item 使用 typedef 定义数据类型的别名
\item auto 的使用场景,delcltype 使用不多简单了解即可
\item 如何自定义更复杂的数据结构,使用 struct
\item 头文件编写以及头文件保护符的使用
\end{itemize}


\subsection{第3章 : 字符串、向量和数组}
\label{sec:orgheadline15}

\subsubsection{章节导读}
\label{sec:orgheadline13}

本章是对第二章节的延伸,介绍更复杂的数据类型string,vector,迭代器及数组,以及它们之间的关系。

建议完成本章的练习:3.4,3.10,3.14,3.20,3.23,3.36,3.40

\subsubsection{核心知识点}
\label{sec:orgheadline14}

本章的核心知识点包含:

\begin{itemize}
\item using 命名空间使用方式
\item 使用标准库类型 string 定义及处理字符串:初始化 string 对象,对字符串进行各种操作,访问对象中的每个字符
\item 使用标准库类型 vector 定义及处理向量:初始化 vector 对象,对 vector 进行添加删除元素等操作
\item \textbf{【难点】} 使用迭代器来循环访问向量/字符串中的元素:迭代器的使用方法,迭代器运算,begin 与 end运算符
\item C++数组的定义、初始化及与指针的关系:数组的定义及初始化,访问和操作数组元素
\item \textbf{【难点】} 了解多维数组,指针及数组地址的理解,多维数组的访问方法
\end{itemize}
\end{document}
